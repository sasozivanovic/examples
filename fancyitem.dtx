% \iffalse meta-comment
% fancyitem.dtx
%% `fancyitem' is a package for XXX
%%
%% Copyright (c) 2020 Saso Zivanovic
%%                    (Sa\v{s}o \v{Z}ivanovi\'{c})
%% saso.zivanovic@guest.arnes.si
%%
%% This work may be distributed and/or modified under the
%% conditions of the LaTeX Project Public License, either version 1.3
%% of this license or (at your option) any later version.
%% The latest version of this license is in
%% 
%%   http://www.latex-project.org/lppl.txt
%%   
%% and version 1.3 or later is part of all distributions of LaTeX
%% version 2005/12/01 or later.
%%
%% This work has the LPPL maintenance status `author-maintained'.
%% 
%% This work consists of the following files:
%% - fancyitem.dtx:
%% - fancyitem.sty:
%% 
%<*driver>
\documentclass[a4paper]{ltxdoc}
\usepackage{doc-preamble}

\usepackage{xparse}
\usepackage[shortlabels]{enumitem}
\usepackage{fancyitem}
% \usepackage{examples2}

\author{Sašo Živanović\footnote{%
    e-mail: \href{mailto:saso.zivanovic@guest.arnes.si}{saso.zivanovic@guest.arnes.si};
    web: \href{http://spj.ff.uni-lj.si/zivanovic/}{http://spj.ff.uni-lj.si/zivanovic/}}}
\title{fancyitem}

\begin{document}

\maketitle

\begin{abstract}
  This package allows the user to redefine the stock |\item|.  It also provides
  an end-of-item hook.
\end{abstract}

\section{Introduction}
\label{sec:introduction}

The idea of the package is that some specialized environments might be better
off with a specialized |\item| command.  For example, within an environment for
linguistic examples, it might be handy to have an |\item| which makes the
judgment mark stick out of the example,
% starts glossing,
and pushes the (parenthesized) language identifier to the right.  Like this:

\begin{tcblisting}{example}
  \NewDocumentCommand\LinguisticItem{d()s}{%           % in the preamble
    \IfValueT{#1}{\AtEndItem{\hfill(#1)}}%
    \PlainItem
    \IfBooleanT{#2}{\mbox{}\llap{*}}%
    \ignorespaces
  }
  \FancyItem[1]{enumerate}{\LinguisticItem}
  \begin{enumerate}[(1),labelsep=1em]                  % in the document
  \item(English) This sentence is grammatical.
  \item(English)* This sentence ungrammatical is.
  \end{enumerate}
\end{tcblisting}

Or perhaps, a listing environment which automatically enables math mode?

\begin{tcblisting}{example}
  \newcommand\MathItem{%                               % in the preamble
    \AtEndItem{$}%
    \PlainItem$%
  }
  \FancyItem[1]{itemize}{\MathItem}
  \begin{itemize}                                      % in the document
  \item (a+1)^2=a^2+2a+1
  \item \exists x\colon Ax\wedge Bx
  \end{itemize}
\end{tcblisting}

Sky is the limit once you have control over your |\item|.


\section{Usage}
\label{sec:usage}

% todo: depth 0

Load the package:

\begin{tcblisting}{example,listing only}
  \usepackage{fancyitem}
\end{tcblisting}

Define an \emph{item macro} (or several) in any way you like, naming it as you
wish:

\begin{tcblisting}{example,listing only}
  \def\myitem{\PlainItem\cmd{\myitem}: }                                    % TeX
  \newcommand\myitemcommand{\PlainItem\cmd{\myitemcommand}: }               % LaTeX
  \NewDocumentCommand\MyItemCommand{}{\PlainItem\cmd{\MyItemCommand}: }     % LaTeX3
\end{tcblisting}

Associate the item macro to an environment at a certain depth:

\begin{tcblisting}{example,listing only}
  \FancyItem[1]{itemize}{\myitem}
  \FancyItem[1]{enumerate}{\myitemcommand}
  \FancyItem[2]{enumerate}{\MyItemCommand}
\end{tcblisting}

Make a list! 

\def\myitem{\PlainItem\cmd{\myitem}: }                                    % TeX
\newcommand\myitemcommand{\PlainItem\cmd{\myitemcommand}: }               % LaTeX
\NewDocumentCommand\MyItemCommand{}{\PlainItem\cmd{\MyItemCommand}: }     % LaTeX3
\FancyItem[1]{itemize}{\myitem}
\FancyItem[1]{enumerate}{\myitemcommand}
\FancyItem[2]{enumerate}{\MyItemCommand}

\begin{tcblisting}{example}
  \begin{enumerate}
  \item enumerated list at depth one
  \item again
    \begin{itemize}
    \item first-level itemize
    \item again
    \end{itemize}
  \item 
    \begin{enumerate}
    \item enumerate at depth two
    \item again
    \end{enumerate}
  \end{enumerate} 
\end{tcblisting}

As a special treat, the package provides a hook into the end of an item.  The
hook is set up by |\AtEndItem|.  It makes most sense within an item macro, like
this:

\begin{tcblisting}{example}
\end{tcblisting}


\DocInput{fancyitem.dtx}

\end{document}


%%% Local Variables:
%%% mode: latex
%%% TeX-master: "fancyitem.dtx"
%%% End:

%</driver>
% \fi
% 
% \section{Implementation}
% \label{sec:fancyitem:implementation}
%
% Package identification and dependencies.
%    \begin{macrocode}
\ProvidesPackage{fancyitem}
\RequirePackage{etoolbox}
%    \end{macrocode}
% 
%
% \subsection{Auxiliary definitions}
% \label{sec:fancyitem:aux}
% 
% The user-defined \emph{item macros} are accessed via control sequences which
% |\fancyitem@itemcs| expands to.  The two parameters are the environment name
% (|#1|) and the environment depth (|#2|).  The depth of 0 is a fallback for
% unregistered depths, see |\FancyItem| below.
%    \begin{macrocode}
\def\fancyitem@itemcs#1#2{fancyitem@item@#1@#2}
%    \end{macrocode}
% 
% The following macro expands to the control sequence of the \emph{end-of-item
% token list} associated to the current environment at the current depth.
%    \begin{macrocode}
\def\fancyitem@endcurritemcs{fancyitem@end@\@currenvir @\fancyitem@dp}
%    \end{macrocode}
% 
% This package is normally meant to be used alongside package |enumitem|, so
% the depth of a list environment is retrieved from |enumitem|'s internal depth
% counters |enitdp@|\meta{environment name}.
%    \begin{macrocode}
\def\fancyitem@dp{%
  \ifcsdef{enitdp@\@currenvir}{%
    \expandafter\the\csname enitdp@\@currenvir\endcsname
  }{}%
}
%    \end{macrocode}
% 
% In the absence of |enumitem|, we provide its alternative names of standard
% \LaTeX\ counters |\@itemdepth| and |\@enumdepth| ourselves.\footnote{If
% |enumitem| is loaded with option |loadonly|, the alternative names are not
% defined and we don't provide them either, assuming that the user knows what
% she is doing.}
%    \begin{macrocode}
\@ifpackageloaded{enumitem}{}{%
  \let\enitdp@itemize\@itemdepth
  \let\enitdp@enumerate\@enumdepth
}
%    \end{macrocode}
%
% \subsection{The core}
% \label{sec:fancyitem:core}
% 
% This is our redefinition of standard |\item|.  If the current environment has
% an associated item macro at the current depth, we use it: we first insert the
% end-of-item token list into the stream, and then execute the user-defined
% macro.  Otherwise, we try to use the fallback item macro for the current
% environment (the ``depth'' 0).  If this fails as well, we fall back to the
% standard \LaTeX\ item (saved in |\PlainItem|).
%    \begin{macrocode}     
\def\fancyitem@item{%
  \ifcsdef{\fancyitem@itemcs{\@currenvir}{\fancyitem@dp}}{%
    \fancyitem@doatenditem
    \@nameuse{\fancyitem@itemcs{\@currenvir}{\fancyitem@dp}}%
  }{%
    \ifcsdef{\fancyitem@itemcs{\@currenvir}{0}}{%
      \fancyitem@doatenditem
      \@nameuse{\fancyitem@itemcs{\@currenvir}{0}}%
    }{%      
      \PlainItem
    }%
  }%
}
%    \end{macrocode}
% 
% If the end-of-item token list for the current environment at the current
% depth exists, insert it into the stream and then clear it.
%    \begin{macrocode}
\def\fancyitem@doatenditem{%
  \ifcsdef{\fancyitem@endcurritemcs}{\fancyitem@doatenditem@}{}}
\def\fancyitem@doatenditem@{%
  \expandafter\the\csname\fancyitem@endcurritemcs\endcsname
  \csname\fancyitem@endcurritemcs\endcsname={}%
}
%    \end{macrocode}
% 
% \subsection{User interface}
% \label{sec:fancyitem:ui}
% 
% \macro{\FancyItem}This macro registers a user-defined item macro (|#3|) for a
% given environment (|#2|) at a given depth (the optional argument |#1|).  If
% the depth is 0 (or omitted), the given item macro will be used as a fallback
% item macro for this environment.
% 
% Note that the user-defined item macro should in principle call the standard
% \LaTeX\ |\item|, now stored in |\PlainItem|.  The item macro may call
% |\AtEndItem|; its argument will be executed at the end of the item.
%    \begin{macrocode}
\newcommand\FancyItem[3][0]{%
%    \end{macrocode}
% Associate the environment--depth dependent control sequence to the given item
% macro.
%    \begin{macrocode}
  \cslet{\fancyitem@itemcs{#2}{#1}}{#3}%
%    \end{macrocode}
% 
% The end-of-item hook for the final item in a list requires special attention.
% While the modified |\item| command inserts the end-of-item token list for
% non-final items, this insertion must be performed by the end of environment
% for the final item.  We use |etoolbox|'s |\AtEndEnvironment| to call
% |\fancyitem@doatenditem| at the end of every registered environment.
% 
% Given that every depth of an environment can have its own item macro,
% |\FancyItem| can be called several times for the same environment.  But
% calling |\fancyitem@doatenditem| once at the end of an environment suffices,
% as this macro deals with depth internally.  Before blindly appending it to
% the end-of-environment hook, we therefore check if it is already there.  We
% do this using |\ifpatchable|.  The test is not bullet-proof, but it should
% suffice.  The test furthermore relies on the internals of |etoolbox| (that
% the end-of-environment hook is stored in |@end@#2@hook|), but the worse that
% can happen if something goes wrong is that |\fancyitem@doatenditem| will get
% executed several times at the end of an environment, which is harmless, as
% the end-of-item token list is cleared after usage.
%    \begin{macrocode}
\expandafter\ifpatchable\expandafter
  {\csname @end@#2@hook\endcsname}{\fancyitem@doatenditem}{}{%
    \AtEndEnvironment{#2}{\fancyitem@doatenditem}%
  }%
}
%    \end{macrocode}
% 
% \macro{\AtEndItem}This macro will usually be called within the item macro, to
% defer some code until the end of item.  After ensuring that the end-of-item
% token list for the current environment at the current depth actually exists
% (if not, it is created), the macro stores the given code into the token list.
%    \begin{macrocode}
\def\AtEndItem#1{%
  \ifcsdef{\fancyitem@endcurritemcs}{}{%
    \expandafter\newtoks\csname\fancyitem@endcurritemcs\endcsname
  }%
  \csname\fancyitem@endcurritemcs\endcsname={#1}%
}
%    \end{macrocode}
% 
% \macro{\PlainItem}Redefine the standard \LaTeX\ |\item|, saving the original
% into |\PlainItem|, so that it can be called by user-defined item macros.
%    \begin{macrocode}
\let\PlainItem\item
\let\item\fancyitem@item      
%    \end{macrocode}
% \endinput
% Local Variables:
% mode: doctex
% TeX-master: t
% End:
