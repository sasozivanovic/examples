\documentclass[a4paper]{ltxdoc}
\usepackage{doc-preamble}

\usepackage{xparse}
\usepackage[shortlabels]{enumitem}
\usepackage{fancyitem}
% \usepackage{examples2}

\author{Sašo Živanović\footnote{%
    e-mail: \href{mailto:saso.zivanovic@guest.arnes.si}{saso.zivanovic@guest.arnes.si};
    web: \href{http://spj.ff.uni-lj.si/zivanovic/}{http://spj.ff.uni-lj.si/zivanovic/}}}
\title{fancyitem}

\begin{document}

\maketitle

\begin{abstract}
  This package allows the user to redefine the stock |\item|.  It also provides
  an end-of-item hook.
\end{abstract}

\section{Introduction}
\label{sec:introduction}

The idea of the package is that some specialized environments might be better
off with a specialized |\item| command.  For example, within an environment for
linguistic examples, it might be handy to have an |\item| which makes the
judgment mark stick out of the example,
% starts glossing,
and pushes the (parenthesized) language identifier to the right.  Like this:

\begin{tcblisting}{example}
  \NewDocumentCommand\LinguisticItem{d()s}{%           % in the preamble
    \IfValueT{#1}{\AtEndItem{\hfill(#1)}}%
    \PlainItem
    \IfBooleanT{#2}{\mbox{}\llap{*}}%
    \ignorespaces
  }
  \FancyItem[1]{enumerate}{\LinguisticItem}
  \begin{enumerate}[(1),labelsep=1em]                  % in the document
  \item(English) This sentence is grammatical.
  \item(English)* This sentence ungrammatical is.
  \end{enumerate}
\end{tcblisting}

Or perhaps, a listing environment which automatically enables math mode?

\begin{tcblisting}{example}
  \newcommand\MathItem{%                               % in the preamble
    \AtEndItem{$}%
    \PlainItem$%
  }
  \FancyItem[1]{itemize}{\MathItem}
  \begin{itemize}                                      % in the document
  \item (a+1)^2=a^2+2a+1
  \item \exists x\colon Ax\wedge Bx
  \end{itemize}
\end{tcblisting}

Sky is the limit once you have control over your |\item|.


\section{Usage}
\label{sec:usage}

% todo: depth 0

Load the package:

\begin{tcblisting}{example,listing only}
  \usepackage{fancyitem}
\end{tcblisting}

Define an \emph{item macro} (or several) in any way you like, naming it as you
wish:

\begin{tcblisting}{example,listing only}
  \def\myitem{\PlainItem\cmd{\myitem}: }                                    % TeX
  \newcommand\myitemcommand{\PlainItem\cmd{\myitemcommand}: }               % LaTeX
  \NewDocumentCommand\MyItemCommand{}{\PlainItem\cmd{\MyItemCommand}: }     % LaTeX3
\end{tcblisting}

Associate the item macro to an environment at a certain depth:

\begin{tcblisting}{example,listing only}
  \FancyItem[1]{itemize}{\myitem}
  \FancyItem[1]{enumerate}{\myitemcommand}
  \FancyItem[2]{enumerate}{\MyItemCommand}
\end{tcblisting}

Make a list! 

\def\myitem{\PlainItem\cmd{\myitem}: }                                    % TeX
\newcommand\myitemcommand{\PlainItem\cmd{\myitemcommand}: }               % LaTeX
\NewDocumentCommand\MyItemCommand{}{\PlainItem\cmd{\MyItemCommand}: }     % LaTeX3
\FancyItem[1]{itemize}{\myitem}
\FancyItem[1]{enumerate}{\myitemcommand}
\FancyItem[2]{enumerate}{\MyItemCommand}

\begin{tcblisting}{example}
  \begin{enumerate}
  \item enumerated list at depth one
  \item again
    \begin{itemize}
    \item first-level itemize
    \item again
    \end{itemize}
  \item 
    \begin{enumerate}
    \item enumerate at depth two
    \item again
    \end{enumerate}
  \end{enumerate} 
\end{tcblisting}

As a special treat, the package provides a hook into the end of an item.  The
hook is set up by |\AtEndItem|.  It makes most sense within an item macro, like
this:

\begin{tcblisting}{example}
\end{tcblisting}


\DocInput{fancyitem.dtx}

\end{document}


%%% Local Variables:
%%% mode: latex
%%% TeX-master: "fancyitem.dtx"
%%% End:
